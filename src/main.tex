%! Author = Paulo
%! Date = 5/13/2021

\documentclass{paulo_cv}
\usepackage{babel}

\setname{Paulo}{Sarrin Cepeda}
\setaddress{Lima/Peru}
\setmobile{+51 931 020 809}
\setmail{paulocesarsarrin@gmail.com}
\setposition{Analista de Ciberseguridad} %ignored for now
\setlinkedinaccount{https://www.linkedin.com/in/paulocsarrinc/} %you can play with color of the template (red is also nice..)
\setgithubaccount{https://github.com/PauloCSC} %you can play with color of the template (red is also nice..)
\setthemecolor{bleudefrance} %you can play with color of the template (red is also nice..)

\begin{document}
%Set variables
%You can add sections, texts, explanations just by copying the style below. Replace the dummy texts "\lipsum[1][x-x]\par" with actual texts.
%Create header
    \headerview
    \vspace{1ex}
%Sections
%
% Summary
    \addblocktext{Resumen}{%
        Alumno del 10mo ciclo de la carrera de Ingeniería de Sistemas en la Universidad Nacional de Ingeniería perteneciente
        al quinto superior de mi promoción universitaria. Amante de la ciberseguridad y hacker ético enfocado en desarrollar
        mis habilidades en el ámbito de la ciberseguridad y ayudar a las empresas a mejorar la securización de sus procesos y
        activos digitales. Me gusta asumir retos y trabajar en equipo.
    }
%
%Education
    \section{Educaci\'on}\label{sec:educacion}
    \datedexperience{Sophos EDR Certified Admin}{Marzo del 2021}
    \explanation{Sophos - Certificaci\'on}
%    \explanationdetail{\coloredbullet\ %
%    Formación práctica para llevar a cabo tareas de Threat Hunting y aprender cómo detener a los atacantes
%    durante un ataque, detectados con Sophos EDR
%    }
    \datedexperience{Cyber Security Foundation Professional Certificate}{Diciembre del 2020}
    \explanation{Certiprof - Certificaci\'on - ID: 54283956}
    \datedexperience{Certified Penetration Testing Engineer}{Mayo del 2020}
    \explanation{Mile2 - Certificaci\'on - ID: 1877000}
    \datedexperience{Universidad Nacional de Ingenier\'ia}{2016-2021}
    \explanation{Bachillerato en Ingenier\'ia de Sistemas}

%
% Experience
    \section{Experiencia}\label{sec:experiencia}
    %
    \datedexperience{Organización de Estados Americanos}{Noviembre del 2020 – Actualidad / Washington - EE.UU.}
    \explanation{Consultor externo en CSIRTAméricas del Programa de ciberseguridad en CICTE}
    \explanationdetail{
        \coloredbullet \ Miembro de la implementación y desarrollo de la herramienta NGEN para la gestión de incidentes
        informáticos en CSIRT’s y CERT’s nacionales/estatales de la red internacional de CSIRTAméricas. \\
        \coloredbullet \ Auditor principal, responsable de la ejecución de servicios de pentesting a CSIRT’s y CERT’s nacionales.}
    %
    \datedexperience{enHacke S.A.C.}{Setiembre del 2019 – Abril del 2021 / Lima - Per\'u}
    \explanation{Analista de ciberseguridad}
    \explanationdetail{
        \coloredbullet \ Responsable (y apoyo) en ejecución de servicios de ciberseguridad a empresas del sector financiero, salud,
        productivo y educativo. Servicios que consisten en evaluar el nivel de ciberseguridad de infraestructura on premise,
        infraestructura on cloud, aplicativos web y móviles.\\
        \coloredbullet \ Apoyo en la ejecución de servicios de análisis forense para empresas víctimas de ciberataques dirigidos como ransomware,
        spear pishing y estafa digital.
    }
    \datedexperience{Ministerio de Trabajo y Promoción del empleo}{Marzo – Setiembre del 2019 / Lima - Per\'u}
    \explanation{Practicante preprofesional en el Área de Estadística e Informática}
    \explanationdetail{
        \coloredbullet \ Apoyo en las labores de Soporte técnico, Administración de servidores y redes de computadoras.
    }
%
%% Logros
    \section{Logros}\label{sec:logros}
    \datedexperience{OAS Cyber Challenge}{Diciembre del 2019 / Cartagena - Colombia}
    \explanationdetail{1° puesto general \cpshalf Concurso de ciberseguridad en modalidad jeopardy entre selecciones nacionales de Brasil, Colombia, Republica Dominicana, Costa Rica y Per\'u}
    \datedexperience{Ibero-American Cybersecurity Challenge }{Diciembre del 2018 / Cartagena - Colombia}
    \explanationdetail{2° puesto general \cpshalf Concurso de ciberseguridad en modalidad jeopardy entre selecciones nacionales de Espa\~na, Colombia, Republica Dominicana, Costa Rica y Per\'u}

%    \newcommand{\extraone}{%
%        \textbf{OAS Cyber Challenge | } \par %replace this part with actual text
%    }
%    %
%    \newcommand{\extratwo}{%
%        \lipsum[1][9-10]\par %replace this part with actual text
%    }
%    %
%    \newcommand{\extrathree}{%
%        \lipsum[1][11-12]%replace this part with actual text
%    }
%    %
%    \newcommand{\listofextras}{\extraone, \extratwo, \extrathree}
%    %
%    \createbullets{\listofextras}
%
% Skills
    \section{Habilidades}\label{sec:habilidades}
    %
    \newcommand{\skillone}{\createskill{Ciberseguridad}{\textbf{\emph{Experienced:}} \ \  Pentesting web \cpshalf Pentesting en red \cpshalf Pentesting Android \ }}
    \newcommand{\skilloned}{\createskill{\ }{ \textbf{\emph{Familiar:}} \ \  Pentesting iOS \cpshalf Analisis Forense \cpshalf Ciberinteligencia}}
    %
    \newcommand{\skilltwo}{\createskill{Seguridad de la informaci\'on}{ ISO 27001 }}
    %
    \newcommand{\skillthree}{\createskill{Lenguajes de programaci\'on}{\textbf{\emph{Experienced:}} \ \  Python \cpshalf NodeJS \cpshalf Bash \ \ \textbf{\emph{Familiar:}} \ \  C/C++ \cpshalf Go \cpshalf Java \cpshalf C\#}}
    %
    \newcommand{\skillfive}{\createskill{Lenguajes}{\textbf{\emph{Nativo:}} \ \  Espa\~nol \ \ \textbf{\emph{Intermedio:}} \ \ English}}
    %
    \createskills{\skillone, \skilloned, \skilltwo, \skillthree, \skillfour, \skillfive}
\end{document}

